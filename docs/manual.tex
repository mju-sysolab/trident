\documentclass[12pt]{article} 
\usepackage{latex8}
%\usepackage{times}
\usepackage{graphicx}

%------------------------------------------------------------------------- 
% take the % away on next line to produce the final camera-ready version 
%\pagestyle{empty}

%-------------------------------------------------------------------------

\begin{document}

\title{Float Compiler Reference Manual }
       
\author{\\
Los Alamos National Laboratory \\ MS B287\\
Los Alamos, NM 87545\\ \\
% For a paper whose authors are all at the same institution, 
% omit the following lines up until the closing ``}''.
% Additional authors and addresses can be added with ``\and'', 
% just like the second author.
%\and
%Second Author\\
%Institution2\\
%First line of institution2 address\\ Second line of institution2 address\\ 
%SecondAuthor@institution2.com\\
}

\maketitle

%\thispagestyle{empty}

%\begin{abstract}
%\end{abstract}



%------------------------------------------------------------------------- 
\section{Introduction}
Trident is an high-level language (HLL) compiler for scientific
algorithms written in C that use float and double data types.  It
produces circuits in reconfigurable logic that exploit the
micro-instruction and pipelined parallelism.  Trident automatically
extracts parallelism and pipelines loop bodies using compiler
optimizations and hardware scheduling techniques.  The Trident
compiler consists of four principal steps shown in
Figure~\ref{fig:overview}.

\begin{figure}[ht]
\centering
\includegraphics[width=.6\textwidth]{figures/overview}
\caption{\label{fig:overview} Four principal steps in the Trident Compiler.}
\end{figure}


The first step in the Trident compilation process is the LLVM C/C++
front-end.  The LLVM (Low Level Virtual Machine) compiler
framework\cite{lattner04llvm} is used to parse input languages, and
produce a low-level platform independent object code.  LLVM optimizes
the object code and then Trident parses the result into its own
intermediate representation (IR).  This approach allows Trident to
concern itself with hardware compilation and leave the parsing and
baseline optimizations in LLVM.

The Trident compiler's second step, IR transformation, accomplishes
several important tasks.  First, operations are converted into
predicated form to allow the creation of hyperblocks.  Next, further
optimizations are performed to remove any unnecessary operations.
Finally, all floating point operations are mapped into a specific
hardware library selected by the user.

Floating point data types, with 32- or 64-bit width, present
challenges in terms of resource allocation, both on- and off-chip.
Floating point modules require significantly more logic blocks on a
reconfigurable chip than small integer operations.  The large input
width also implies that more memory bandwidth is required for floating
point arrays. Scheduling is further complicated by the interaction
with resource allocation.  The operations will be scheduled with one
of four different scheduling algorithms: ASAP, ALAP, Force Directed,
and for pipelined loops, Iterative Modulo scheduling.

The final step synthesizes the results of the scheduler into Register
Transfer Level (RTL) VHDL.  Hierarchy is utilized in the hardware
description in order to preserve modularity.  The top-level circuit
contains subcircuits for each block in the control flow graph input
and also a single register file.  The register file is shared by all
block subcircuits.  Each block subcircuit contains a state machine and
a datapath subcircuit.  The state machine is determined by the
initiation interval of the design, and it controls the timing of the
block's datapath.  The datapath implements the logic needed to
represent the flow of data through all of the operations found in the
control flow graph.  It contains operators, predicate logic, local
registers and wires that connect all of the components.  All of these
elements combine to produce a highly optimized application specific
circuit.

Trident provides an open framework for exploration of floating-point
libraries and computation in FPGAs.  The Trident framework allows for
user optimizations to be added at several levels of abstraction.
Users can trade-off clock speed and latency through their selection of
different floating-point libraries and optimizations.  It is also
possible for users to import their own floating point libraries.  New
hardware platforms may be added by defining new interface description
files and producing the code to tie the design to the description
interface.


%------------------------------------------------------------------------- 
%\section{Float Compiler Organization}

You can add, delete or change the order of compiler passes in the Java portion of the compiler in the compiler/fp/Compile.java file.

\begin{figure}[ht]
%\includegraphics[height=8 in]{org-diag.ps}
\caption{\label{org-diag} Float Compiler Organization}
\vspace{-.25in}
\end{figure}

\section{Source Installation}
There are several required elements to use the srouce installation.

\begin{itemize}
\item LLVM - obtain version 1.5 from www.llvm.org
\item LLVM cfrontend - obtain version 1.5 from www.llvm.org
\item ant - obtain 1.6.X from ant.apach.org, 1.7.X is untested.
\item antlr.jar - obtain 2.7.3 or later from www.antlr.org
\item java-getopt.jar - obtain version 1.11 or later from \\
http://www.urbanophile.com/arenn/hacking/download.html
\item Java - obtain from java.sun.com (tested with 1.4.2)
\item Python - obtain from www.python.org
\item gcc, unix environment, etc.
\item Graphviz (dotty and friends) from www.graphviz.org can be useful, but is not required.
\end{itemize}

First obtain LLVM and compile LLVM.  The following works on RHEL 3.0u5
with tcsh, and detailed instructions are available on www.llvm.org.
If you prefer bash, use the proper commands for setting variables
(e.g., export BLAH=SOME\_VALUE and export
PATH=\$PATH:NEW\_PATH\_VALUE, etc.).

\begin{verbatim}
> tar xzf llvm-1.5.tar.gz
> tar xzf cfrontend-1.5.i686-redhat-linux-gnu.tar.gz

#
# must have java
#
> which java
>
> setenv TOP `pwd`

> cd cfrontend/x86/
> ./fixheaders
> cd ../..

> set path = ( $path $TOP/cfrontend/x86/llvm-gcc/bin/ )

> mkdir llvm-obj
> cd llvm-obj

> ../llvm/configure
> make
# wait 

> cd ..
> set path = ( $path $TOP/llvm-obj/Debug/bin/ )

\end{verbatim}

Now LLVM should be built.  This will build the Debug version of LLVM
and anything else may not work without changing some of the scripts
provided.

Next, unpack the Trident files and execute the
following:\footnote{This is a continuation of the above and uses the
  same TOP variable.  It could be modified to support a different
  location than parallel with LLVM.}

\begin{verbatim}
> tar xzf dist.tgz

# build llv
# 
> cd trident/llv-src
> ./mkconf.sh --src=$TOP/llvm
> ./configure --with-llvmsrc=$TOP/llvm/ --with-llvmobj=$TOP/llvm-obj/

> make
> cp Debug/bin/llv ../bin
> strip ../bin/llv
> chmod a+x ../bin/llv

> cd ..

# copy jars to $TRIDENT_TOP/lib
#
> cp somewhere/antlr.jar lib/
> cp somewhere/java-getopt.jar lib/
\end{verbatim}

Next, compile Trident.  This requires ant, python, java and the jars
just added to the above lib directory.  Modify your Java classpath to
include the jars.

\begin{verbatim}
> cd trident-src/fp
> ant dist
> cd ../..
> cp trident-src/dist/lib/Trident.jar lib/

> set path = ($path $TOP/trident/bin )

> setenv TRIDENT_TOP $TOP/trident
\end{verbatim}

Now, you should have all the tools necessary to run Trident.  Test the 
compiler using the following:

\begin{verbatim}
cd examples/conditional
tcc -t vhdl conditional_a.c run

less conditional_a.vhd
\end{verbatim}

This will generate a VHDL file for the conditional\_a.c example.  The
output of the compiler is extremely verbose and needs to be cleaned up.
Serious errors will cause the compiler to exit, the many warnings can
usually be safely ignored.



\section{Using the Trident Compiler}
Assuming that everything worked correctly in the previous section, here we explain some
of the options and how to use the Trident compiler.

\subsection{Trident}

The compiler is made up of three separate parts, which are all
integrated in the tcc script.  The first part is the LLVM cfront-end,
which reads the input C file and generates LLVM bytecode.  The next
part of the compiler is llv.  llv takes the generated LLVM bytecode,
optimizes it in LLVM and generates parseable text file, which has the
extension .llv.  This is still a target independent representation.
The .llv file is parsed by the java class fp.Compile, undergoes
further optimization and finally is targeted to a particular floating
point library.  The output result is VHDL.


An example of using tcc:

\begin{verbatim}
tcc -t vhdl if_a.c run
\end{verbatim}

The -t option selects the output type and the next two options if\_a.c
and run are the input file and the function to be compiled.  Trident
will only compile a single function from an input file, the function
cannot have any parameters and cannot return any results.  For
example:

\begin{verbatim}
extern int a,b,c,d;

void run() { 
  if (a < b) {
    a = c;
  } else {
    a = d;
  }
}
\end{verbatim}

The run function does not have any input parameters.  However,
variables that are declared as extern can be used as inputs or outputs
(or both.)  Any variables that are declared as extern will get
registers that can be read and written.  Other variables may not be
able to be seen outside of the datapath.


Options to tcc:

\begin{verbatim}
Usage: tcc [-h] [--llv=/path/to/llv] [compiler-options] input_file function_name

Version 0.5

View compiler options by using the command: 'tcc -h' 
Environment variables LLVMGCCDIR and CLASSPATH must be set correctly
Executables llv and java must be in the user's path
\end{verbatim}


Most options to tcc pass to the backend Trident compiler as does the -h options.  The
backend compiler has the following options:

\begin{verbatim}
Compile [Options] input_file function_name

Short Options:
-a filename     : Filename of architecture (hardware) description
-b              : Generate Testbench -- experimental
-h              : Get help -- good luck!
-l libname      : Specify library, supporting quixilica, aa_fplib, 
-t format       : Target output.  format can be set to "dot" or "vhdl"

Long Options:
--sched=sched_list
        sched_list is comma-separated list which can include
        asap, alap, or fd, and optionally nomod or mod

        It is case insensitive.  You must choose either asap, alap, or force
        directed.  It is default to run modulo scheduling on all loop blocks,
        but to turn that off, you can add "nomod" within the string.  For
        example:

        --sched=fd,nomod

        You can also, specify "mod", but that is the default.

--sched.options=option_list
        option_list is a comma-separated list which can include
        considerpreds (tell schedule not to ignore predicates)
        dontpack (don't pack multiple less than 1-clock tick instructions within a single cycle)
        conserve_area (conserve area by only using as many logic units as 
                necessary so as to not slow down execution of the design)
        fd_maxtrycnt (tell the compiler how many times to attempt force-directed 
                scheduling before giving up)
        ms_maxtrycnt (how many times to attempt modulo scheduling)
        cyclelength (set the length of a clock tick (the default is 1.0))

        The options can be set either by using the long opt, "sched.options" to set several at
        once (like this:

        --sched.options=considerpreds,dontpack,conserve_area

        or they can be set individually by saying:

        --sched.options.considerpreds
\end{verbatim}

Several options may not be very useful and they also may not be very
well tested.  \textbf{-a} is for specifying a different hardware target.
Currently only one hardware target is truly supported -- specifying
other files here is not well tested.  \textbf{-b} generates a skeleton VHDL
testbench for the top-level cell.  It does this by examining the cell,
instancing it and inserting some reset statements.  Treat this option
as experimental -- if it works for you, good -- if it does not, you
were warned.  \textbf{-l} is for specifying additional libraries.  This option
is dynamic and supports the floating point libraries it has been told
about.  aa\_fplib is the library I have supplied. \textbf{-t} specifies the
output format, vhdl or dot.  vhdl produces a vhdl netlist and dot is a
schematic view of that netlist.  Please be aware that hierarchical
netlists quickly exceed dotty's (graphviz) ability to render dot
files.

The long options are described above.  Please note that by default
loops are modulo scheduled and non-loop blocks are scheduled using a
force-directed algorithm.  Other options modify how particulars within
the scheduling algorithms.

In addition to Dot circuit representations, Trident currently creates
some debug information in Dot format.  The
\textit{cfile}\_\textit{function}.dot file is the final optimized
version of the function before synthesis. The synthesized version is
called syn\_\textit{cfile}.dot.  Other dot files represent the data
flow graph of individual hyperblocks.


\subsection{Simulation}

Trident generates a single VHDL file representing the compiled design.
Although, there is just one file, the file contains multiple
heirarchical VHDL Design Units and contains the elements shown in
Figure~\ref{fig:abs_circuit}.  To allow for correct compilation, the
circuit elements are inserted in the VHDL file from the lowest level
of heirarchy (leaves) to the highest level (top).  At the top level,
all of the inputs and outputs are available as registers as well as
\textit{start} and \textit{reset} signals.

\begin{figure}[ht]
\centering
\includegraphics[width=.6\textwidth]{figures/abstract_circuit}
\caption{\label{fig:abs_circuit} Circuit Elements found in the Trident compiler.}
\end{figure}

%
% Compilation and Libraries

Compilation of Trident generated designs usually requires several
support libraries.  It is best to compile the libraries first and then
compile the design to verify that all circuit objects have been
correctly instanced.  A floating-point circuit will require a
floating-point library as well as the trident support library.  The
floating-point library should provide its own means of compilation.
The trident support library has several operations that are not
normally included in most available floating-point libraries.  This
includes casts and a few other functions(e.g., float to int, double to
long, int to float, long to double, float to double, double to float,
fpabs, and fpinv).\footnote{Casts are ugly operations that are easy to
  imply when mixing floats with integer types.}

An example of what compilation may look like:

\begin{verbatim}
> vlib work
> vmap work work
> vmap fplib /my/path/to/fplib/fplib
> vcom -93 example.vhd tb_example.vhd
\end{verbatim}

%
% Test benches.

The compiler does have the ability to generate a skeleton VHDL
testbench for generated designs.  This testbench does not actually
exercise the circuit, but instances it and produces process that
can be used to provide input and test the results.  The generated
testbench saves the user the effort of writing the testbench from
scratch.

%
% expand
The synthesized circuit has a particular model of execution.  The
testbench must follow this model to ensure proper circuit function.
The circuit expects all of the inputs to be available before the start
signal is toggled.  Reset should be toggled before start and will
reset any values written to registers before reset has been toggled.
The circuit design does not eliminate the possibility of modifying the
registers during operation, however, this may result in incorrect
operation.


%\subsection{Final Synthesis}
%\section{Building the Float Compiler}

The following are needed to build and run the float compiler:
\begin{itemize}
\item CVS checkout of float/compiler
\item LLVM version 1.3
\item Java compiler (we use jikes)
\item JVM (we use j2sdk-1.4.2\_08)
\item Ant (we use Ant version 1.6.2)
\item Antlr  
\item Dotty
\item Python
\end{itemize}

The float compiler uses LLVM as a front end, so LLVM must be installed on your system.  It is available from http://llvm.cs.uiuc.edu.  Currently the float compiler is using LLVM version 1.3.  We installed an optimized version (which they call "Release"), because if you install the debug version, you are required to use GCC 3.4.0 and we wanted to be more flexible. Note: we installed llvm in /n/projects/rcc/float/llvm-1.3.  Once LLVM is installed, you need to set up your environment appropriately.  We execute the following from a script called llvm.csh located in /n/projects/rcc/float/llvm-1.3, so you can just "source" that script.

\begin{verbatim}

setenv LLVM_TOP /n/projects/rcc/float/llvm-1.3/
setenv LLVMGCCDIR $LLVM_TOP/cfrontend/x86/llvm-gcc

alias llvmgcc $LLVMGCCDIR/bin/gcc
alias llvmg++ $LLVMGCCDIR/bin/g++

setenv LLVM_LIB_SEARCH_PATH $LLVMGCCDIR/bytecode-libs
set path= ( $path $LLVM_TOP/obj/tools/Release/ )

\end{verbatim}


Next, make sure you check out (from CVS) float/compiler.  

For llv, it is a project of LLVM and the source is in float/compiler/llvm-1.3.  It must be compiled according to its Makefile and then the llv binary can be put in your llvm installation's obj/tools/Release, where the llvm binaries are also.

JAVA\_HOME should be set to the location of your JVM.   We use /packages/lib/j2sdk-1.4.2\_08.

CLASS\_PATH should be set to ".".

Your path should include your ant (we use /packages/lib/apache-ant-1.6.2), your JAVA\_HOME/bin, your Java compiler, your "dotty" and "python" (which for us are in /packages/bin).

Now, to build the Java part of the compiler (fp) you can either build from the compiler directory or from the fp directory.   From the compiler directory, the ant targets are "init-fp", "compile-fp"and "clean-fp".  This builds a directory called "build" at the same level as "fp" and "examples", where the source files are copied and the compiled files will also be put.  You can also build the compiler from the fp directory.  There the ant targets are "init", "compile" and "clean".
 





%\section{Running the Tests and Examples}

The float compiler has both tests and examples.  Tests are unit tests that are written as Java classes (starting with TEST) and located within the fp directory hierarchy in subdirectories called "test".  Examples are sample programs that are located in the example directory hierarchy in subdirectories that comprise "suites" of examples.  

To compile and run tests from the compiler directory, the ant targets are "init-tests", "compile-tests", "run-tests", "test-usage" and "clean-tests".  To compile and run the tests from the "fp" dirctory,the ant targets are "init-tests", "run-tests", "test-usage" and "clean-tests".  These targets, by default, do work for all the tests.  If you want to limit the tests, you can give ant the -D option.  "ant -Dstart.path=util run-tests". 

To compile and run the examples from the compiler directory, the ant targets are "init-examples", "compile-examples", "run-examples", "example-usage" and "clean-examples".  The ant targets are the same if you are compiling and running the examples from the "examples" directory.  You can limit the suites of examples run by giving ant the -D option.  "ant -Dstart.dir=photon run-examples".




%------------------------------------------------------------------------- 
%\nocite{ex1,ex2}
\bibliographystyle{plain}
\bibliography{manual}

\end{document}

